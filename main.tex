\documentclass[12pt,]{scrartcl}
\usepackage{lmodern}
\usepackage{amssymb,amsmath}
\usepackage{ifxetex,ifluatex}
\usepackage{fixltx2e} % provides \textsubscript
\ifnum 0\ifxetex 1\fi\ifluatex 1\fi=0 % if pdftex
  \usepackage[T1]{fontenc}
  \usepackage[utf8]{inputenc}
\else % if luatex or xelatex
  \ifxetex
    \usepackage{mathspec}
  \else
    \usepackage{fontspec}
  \fi
  \defaultfontfeatures{Ligatures=TeX,Scale=MatchLowercase}
\fi
% use upquote if available, for straight quotes in verbatim environments
\IfFileExists{upquote.sty}{\usepackage{upquote}}{}
% use microtype if available
\IfFileExists{microtype.sty}{%
\usepackage[]{microtype}
\UseMicrotypeSet[protrusion]{basicmath} % disable protrusion for tt fonts
}{}
\PassOptionsToPackage{hyphens}{url} % url is loaded by hyperref
\usepackage[unicode=true]{hyperref}
\hypersetup{
            pdfborder={0 0 0},
            breaklinks=true}
\urlstyle{same}  % don't use monospace font for urls
\usepackage[margin=1in]{geometry}
\IfFileExists{parskip.sty}{%
\usepackage{parskip}
}{% else
\setlength{\parindent}{0pt}
\setlength{\parskip}{6pt plus 2pt minus 1pt}
}
\setlength{\emergencystretch}{3em}  % prevent overfull lines
\providecommand{\tightlist}{%
  \setlength{\itemsep}{0pt}\setlength{\parskip}{0pt}}
\setcounter{secnumdepth}{0}
%\renewcommand{\baselinestretch}{1.5} % Added by Nickos
% Redefines (sub)paragraphs to behave more like sections
\ifx\paragraph\undefined\else
\let\oldparagraph\paragraph
\renewcommand{\paragraph}[1]{\oldparagraph{#1}\mbox{}}
\fi
\ifx\subparagraph\undefined\else
\let\oldsubparagraph\subparagraph
\renewcommand{\subparagraph}[1]{\oldsubparagraph{#1}\mbox{}}
\fi

% set default figure placement to htbp
\makeatletter
\def\fps@figure{htbp}
\makeatother

\defaultfontfeatures{Scale=MatchLowercase,Ligatures={TeX}}
\setmainfont{Liberation Serif}
\setsansfont{Liberation Serif}
\setmonofont[SmallCapsFont={Liberation Serif}]{Liberation Serif}
\usepackage{hyperref}
\usepackage{multirow}
\usepackage{xcolor}
\definecolor{cvlinkcolor}{RGB}{6,125,233}
\hypersetup{colorlinks,linkcolor=cvlinkcolor,citecolor=cvlinkcolor,urlcolor=cvlinkcolor,bookmarksnumbered,bookmarks=false,pdfinfo={Title={Curriculum Vitae},Author={Nikolaos Tourvas},Creator={Lighton Phiri},Subject={Curriculum Vitae},Keywords={Curriculum Vitae},CreationDate={D:20150513232724}}}
\renewcommand\labelenumi{[\theenumi]}

\date{}

\begin{document}

\begin{table}[h]
%%%\centering
{\def\arraystretch{1.0}\tabcolsep=0pt
\begin{tabular}{p{0.42\linewidth}p{0.05\linewidth}p{0.35\linewidth}}

  \multirow{1}{*}{\LARGE \textbf{Νικόλαος Τουρβάς}} &  &  \\
  
  & & \\
  
  Ιωάννου Χρυσοστόμου 13 & \multicolumn{1}{r}{Researchgate:\;\;} & \multicolumn{1}{l}{\url{researchgate.net/profile/Nikolaos_Tourvas}} \\
  
  Θεσσαλονίκη, 54639 & \multicolumn{1}{r}{ORCID iD:\;\;} & \multicolumn{1}{l}{\href{https://orcid.org/0000-0002-0476-4468}{orcid.org/0000-0002-0476-4468}} \\
  
  Τηλέφωνο: $+$30 6957686563 & \multicolumn{1}{r}{Github:\;\;} & \multicolumn{1}{l}{\href{https://github.com/nikostourvas}{https://github.com/nikostourvas}} \\
  
  Email: {\href{nikostourvas@gmail.com}{nikostourvas@gmail.com}} 
\end{tabular}%
}
\end{table}


\subsection{Εκπαίδευση}\label{Εκπαίδευση}
\vspace{-3mm}
\textbf{Σεπτ 2012 -- Απρ 2018: Πτυχίο Δασολογίας και Φυσικού Περιβάλλοντος, Αριστοτέλειο Πανεπιστήμιο Θεσσαλονίκης, Βαθμός 7,86 - Λίαν Καλώς}
\begin{itemize}
%\vspace{-3mm}
\setlength\itemsep{-0.5em}
\item \textit{Πτυχιακή Διατριβή}: Συγκριτική ανάλυση μοριακής γενετικής ποικιλότητας
ώριμων ατόμων και αναγέννησης ελάτης στα πλαίσια
προγράμματος γενετικής παρακολούθησης
\vspace{2mm}
\newline
\textit{Επιβλέπων καθηγητής}: καθηγητής Φίλιππος Α. Αραβανόπουλος
\end{itemize}

\vspace{-3mm}

\subsection{Εργασιακή Εμπειρία}\label{Εμπειρία}
\vspace{-3mm}
\textbf{Νοε 2016 - Τώρα: Συμμετοχή κατά διαστήματα στα ευρωπαϊκά έργα "Informed"} (Ενοποιημένη Έρευνα για την Αντοχή και Διαχείριση Δασών στη Μεσόγειο) (12 μήνες) και \textbf{"LifeGenMon"} (Ευρωπαϊκό σύστημα Γενετικής Παρακολούθησης των Δασών) (4 μήνες) - Εργαστήριο Δασικής Γενετικής και Βελτίωσης Δασοπονικών Ειδών - ΑΠΘ, Θεσσαλονίκη 

\vspace{3mm}
\begin{itemize}
\vspace{-3mm}
\setlength\itemsep{-0.6em}
\item Δειγματοληψία φυτικού υλικού, Εκχύλιση DNA, PCR, Στατιστική ανάλυση μοριακών δεδομένων SSR πληθυσμών υβριδογενούς ελάτης
\item Συλλογή δεδομένων πεδίου, Συλλογή δεδομένων μέσω συνεντεύξεων, Στατιστική ανάλυση μοριακών δεδομένων (SSR και ISSR) δασοπονικών ειδών 
\vspace{2mm}
\newline
\textit{Επιστημονικά υπεύθυνος}: καθηγητής Φίλιππος Α. Αραβανόπουλος
\end{itemize}

\vspace{3mm}
\textbf{Απρ - Μάι 2016: Πρακτική άσκηση - Ινστιτούτο Εφαρμοσμένων Βιοεπιστημών - ΕΚΕΤΑ, Θεσσαλονίκη}


\vspace{3mm}
\begin{itemize}
\vspace{-3mm}
\setlength\itemsep{-0.6em}
\item Γενετική ανάλυση ποικιλιών μήλου (Εκχύλιση DNA, PCR) 
\item Εκχύλιση DNA από παλαιωμένη ξυλεία πεύκης και ταξινομικός προσδιορισμός βάσει της χλωροπλαστικής αλληλουχίας \textit{trnL} με τη μέθοδο HRM
%\end{itemize}
%\vspace{-2mm}
%\hspace{11mm}
\vspace{2mm}
\newline
\textit{Επιβλέπων, EKETA}: Δρ. Παναγιώτης Μαδέσης
\newline
%\hspace{11mm}
\textit{Επιβλέπων, ΑΠΘ}: καθηγητής Φίλιππος Α. Αραβανόπουλος
\end{itemize}

\subsection{Εργαστηριακές δεξιότητες}\label{lab}
\begin{itemize}
\vspace{-3mm}
\setlength\itemsep{-0.6em}
\item Εκχύλιση DNA (CTAB, Extraction kits)
\item Ποσοτικοποίηση νουκλεικών οξέων με φασματοφωτόμετρο τύπου Nanodrop
\item Multiplex / touchdown / gradient / nested PCR, qPCR - HRM
\item Γενοτύπηση μοριακών δεικτών (GeneMapper)
\end{itemize}

\subsection{Δεξιότητες ανάλυσης γενετικών δεδομένων}\label{genetics}
\vspace{-3mm}
Λογισμικό: GenAlEx, FSTAT, Genepop, Arlequin, MEGA, POPTREE2, NeEstimator, Bottleneck, R (πολλαπλά πακέτα), κ.α.
\begin{itemize}
\vspace{-1mm}
\setlength\itemsep{-0.6em}
\item Βασικές παράμετροι ποικιλότητας
\item Γενετική διαφοροποίηση: AMOVA, Ομαδοποίηση βάσει πολυμεταβλητών αναλύσεων (PCA, PCoA, CA, DAPC)
\item Φυλογενετική ανάλυση: UPGMA, Neighbor-Joining
\item Εκτίμηση δραστικού μεγέθους πληθυσμού, γενετικών στενωπών, μηδενικών αλληλομόρφων 
\item Mantel test:  Isolation By Distance
\item Επαναληψιμότητα - διαμοιρασμός στατιστικών αναλύσεων: Συγγραφή R Notebooks
%\item FST outliers
\end{itemize}

\subsection{Δεξιότητες χρήσης Η/Υ}\label{it}
\begin{itemize}
\vspace{-3mm}
\setlength\itemsep{-0.6em}
\item Προγραμματισμός: γλώσσα R (βασικό επίπεδο)
%- δείγματα στο \href{https://github.com/nikostourvas}{github})
\item Λειτουργικά συστήματα: Windows (συμπ. Microsoft Office), Linux
\item Κατασκευή γραφημάτων: ggplot2 - R
\item Συστήματα στοιχειοθεσίας: Markdown
%\item Συστημα ελέγχου εκδόσεων (git - Github)
\item ArcGIS, QGIS (κατασκευή χάρτη, παρουσίαση γενετικών δεδομένων)
\item Χωρική ανάλυση οικοσυστημικών υπηρεσιών (πλατφόρμα InVEST)
\end{itemize}

\subsection{Επιμόρφωση}\label{courses}
\begin{itemize}
\vspace{-3mm}
\setlength\itemsep{-0.6em}
\item Εντατικό μάθημα βραχείας διάρκειας: "Concepts, Methods and Tools for an Integrated Approach of Resilience in Mediterranean Forests", Zaragoza (Spain), 12-16 December 2016
%\item Διεθνές εκπαιδευτικό εργαστήριο (workshop): "Informed 2nd Modelling workshop", Madrid (Spain), 23-25 November 2016
\end{itemize}

\subsection{Γλωσσικές Δεξιότητες}\label{Γλώσσες}
\begin{itemize}
\vspace{-3mm}
% * <nikostourvas@gmail.com> 2018-08-12T12:16:45.474Z:
%
% ^.
\setlength\itemsep{-0.6em}
\item \textbf{Αγγλική γλώσσα} - Άριστη γνώση - C2 - University of Michigan

\item \textbf{Γερμανική γλώσσα} - Καλή γνώση - Β2 - Goethe Institut
\end{itemize}

%\newpage


\subsection{Δημοσιεύσεις}\label{publications}

\subsubsection{Επιστημονικά περιοδικά με συντελεστή απήχησης}\label{journals}
\vspace{-3mm}
\begin{enumerate}

\leftskip-0.07in
\item Ganopoulos, I., \textbf{Tourvas, N.}, Xanthopoulou, A., Aravanopoulos, F. A., Avramidou, E. V., Zambounis, A., Tsaftaris, A., Madesis, P., Sotiropoulos, T. and Koutinas, N. (2018) ‘Phenotypic and molecular characterization of apple (\textit{Malus} x \textit{domestica} Borkh) genetic resources in Greece’, \textit{Scientia Agricola}, 75(6), pp. 509–518. doi: \href{http://dx.doi.org/10.1590/1678-992x-2016-0499}{10.1590/1678-992x-2016-0499}


\end{enumerate}


\subsubsection{Πρακτικά Συνεδρίων}\label{conferences}

\vspace{-3mm}
\begin{enumerate}
\leftskip-0.13in
	\setcounter{enumi}{1}
	
\leftskip-0.07in  

\leftskip-0.07in  
\item {\textbf{Tourvas, N.} and Aravanopoulos, F. A. (2018) ‘Genetic aspects of reforestation practices for the maximization of forest ecosystem resilience’, in \textit{International Conference Reforestation Challenges.} Belgrade, Serbia, p. 82.}

\leftskip-0.07in  
\item {Aravanopoulos, F. A., Westergren, M., \textbf{Tourvas, N.}, Fussi, B., Finzgar, D., Bajc, M., Kavaliauskas, D., Malliarou, E., Kiourtsis, F. and Kraigher, H. (2018) ‘Genetic monitoring of hybridogenus populations: an analysis of \textit{Abies borisii-regis}, a hybrid between \textit{Abies alba }and the Greek endemic \textit{Abies cephalonica}’, in \textit{AForGeN Alpine Forest Genomics Network.} Kranjska Gora, Slovenia.}

\item {\textbf{Tourvas, N.} and Aravanopoulos, F. A. (2018) ‘Genetic analysis and genetic monitoring of fir at different spatial scales’, in \textit{5th Students Conference: Organic Farming, Remote Sensing of Environment.} Thessaloniki: Faculty of Agriculture, Forestry and Natural Environment.

\leftskip-0.07in  
\item {Aravanopoulos, F. A., Avramidou, E. V., Malliarou, E., \textbf{Tourvas, N.}, Kotina, V., Korompoki, I., Ganopoulos, I., Alizoti, P., Barbas, E., Bekiaroglou, P., Hasilidis, P., Roussakis, G., Kiourtsis, F. and Fragiskakis, N. (2017) ‘Forest genetic monitoring (FGM) applied: first results from two FGM sites in Greece’, in \textit{Forest Genetics 2017: Health and Productivity under Changing Environments. A Joint Meeting of WFGA and CFGA,} University of Alberta, Edmonton.}

\leftskip-0.07in  
\item {Aravanopoulos, F. Α., \textbf{Tourvas, N.}, Avramidou., E., Alizoti, P., Barbas, E. (2017). The European research project INFORMED and the research in Greece on the resilience of Mediterranean forests. In: Proc. \textit{18th Pan-Hellenic Forest Science Society Conference,} Edessa, pp. 429-433}
 
\leftskip-0.07in  
\item {\textbf{Τουρβάς, Ν.}, Μαλλιαρού, E., Αβραμίδου, Ε. Β. Αραβανόπουλος, Φ. Α. (2017) ‘Συγκριτική ανάλυση μοριακής γενετικής ποικιλότητας ώριμων ατόμων και αναγέννησης ελάτης στα πλαίσια προγράμματος γενετικής παρακολούθησης’, Πρακτικά \textit{18ο Πανελλήνιο Δασολογικό Συνέδριο.} Έδεσσα.}

\leftskip-0.07in  
\item {\textbf{Tourvas, N.}, Malliarou, E., Avramidou, E. V. and Aravanopoulos, F. A. (2016) ‘Genetic Monitoring in fir: Initial results from comparative analysis of molecular genetic diversity of adult and regeneration individuals’, in \textit{8th Hellenic Ecological Society Conference.} Thessaloniki.}

\leftskip-0.07in  
\item {Aravanopoulos, F. A., Avramidou, E. V., Malliarou, E., \textbf{Tourvas, N.}, Ganopoulos, I., Alizoti, P., Barbas, E., Bekiaroglou, P., Hasilidis, P., Roussakis, G., Kiourtsis, F., Fragiskakis, N., Westergren, M., Fussi, B., Konnert, M. and Kraigher, H. (2016) ‘Life for forest genetic monitoring: an opportunity to monitor marginal forest tree populations in Greece’, in \textit{COST Action FP1202 MaP-FGR MC final meeting and COST FP1202, EUFORGEN, IUFRO WG 20213 joint final Conference on ‘Marginal and peripheral tree populations: a key genetic resource for European forests’,} Arezzo, Italy.}

\leftskip-0.07in  
\item {\textbf{Τουρβάς, Ν.}, Αβραμίδου, Ε. Β., Αραβανόπουλος Φ. Α. (2016) ‘Πρώτες συγκριτικές αναλύσεις ατόμων ελάτης (\textit{Abies borisii-regis}
) σε διαφορετικό στάδιο ανάπτυξης με τη χρήση μικροδορυφόρων SSR’, Πρακτικά \textit{16ο Συνέδριο της Ελληνικής Επιστημονικής Εταιρείας Γενετικής Βελτίωσης των Φυτών.} Φλώρινα.}

\end{enumerate}


\subsubsection{Κεφάλαια βιβλίων διεθνών εκδόσεων}\label{book-chapters}

\begin{enumerate}
	\setcounter{enumi}{10}
\leftskip-0.07in
\vspace{-3mm}

\item Aravanopoulos F.A., Alizoti P.G., \textbf{Tourvas N.}, Malliarou E., Avramidou E.V., Korompoki I.V., Kotina V.M., Barbas V. ,  Farsakoglou A.M. (2018). Overview on forest genetic monitoring including case studies on FGM for two species from Greece. In: Sijacic-Nikolic M., Milovanovic J. & M. Nonic (eds). \textit{Forests of Southeast Europe Under a Changing Climate: Conservation of Genetic Resources}. Springer (in press).

\newpage

\item Aravanopoulos F.A., Alizoti P.G., Farsakoglou A.M., Malliarou E., Avramidou E.V. , \textbf{Tourvas N.}  (2018). State of biodiversity and forest genetic resources in Greece in relation to conservation. In: Sijacic-Nikolic M., Milovanovic J. & M. Nonic (eds). \textit{Forests of Southeast Europe Under a Changing Climate: Conservation of Genetic Resources}. Springer (in press). 

\end{enumerate}

\subsubsection{Υπό συγγραφή}\label{book-chapters}

\begin{enumerate}
	\setcounter{enumi}{12}
\leftskip-0.07in
\vspace{-3mm}

\item Westergren, M., \textbf{Tourvas, N.}, Finzgar, D., Kavaliauskas, D., Bajc, M., Malliarou, E., Fussi, B., Kraigher, H., Aravanopoulos, F. A. 'Estimation of sample size for accurate estimation of genetic diversity'

\end{enumerate}

%\newpage

%\subsection{References}\label{references}

%Available on request.

\end{document}