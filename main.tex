\documentclass[12pt,]{scrartcl}
\usepackage{lmodern}
\usepackage{amssymb,amsmath}
\usepackage{ifxetex,ifluatex}
\usepackage{fixltx2e} % provides \textsubscript
\ifnum 0\ifxetex 1\fi\ifluatex 1\fi=0 % if pdftex
  \usepackage[T1]{fontenc}
  \usepackage[utf8]{inputenc}
\else % if luatex or xelatex
  \ifxetex
    \usepackage{mathspec}
  \else
    \usepackage{fontspec}
  \fi
  \defaultfontfeatures{Ligatures=TeX,Scale=MatchLowercase}
\fi
% use upquote if available, for straight quotes in verbatim environments
\IfFileExists{upquote.sty}{\usepackage{upquote}}{}
% use microtype if available
\IfFileExists{microtype.sty}{%
\usepackage[]{microtype}
\UseMicrotypeSet[protrusion]{basicmath} % disable protrusion for tt fonts
}{}
\PassOptionsToPackage{hyphens}{url} % url is loaded by hyperrefhttps://www.overleaf.com/project/5b9ae091c59d5e5d057cc838
\usepackage[unicode=true]{hyperref}
\hypersetup{
            pdfborder={0 0 0},
            breaklinks=true}
\urlstyle{same}  % don't use monospace font for urls
\usepackage[margin=1in]{geometry}
\IfFileExists{parskip.sty}{%
\usepackage{parskip}
}{% else
\setlength{\parindent}{0pt}
\setlength{\parskip}{6pt plus 2pt minus 1pt}
}
\setlength{\emergencystretch}{3em}  % prevent overfull lines
\providecommand{\tightlist}{%
  \setlength{\itemsep}{0pt}\setlength{\parskip}{0pt}}
\setcounter{secnumdepth}{0}
%\renewcommand{\baselinestretch}{1.5} % Added by Nickos
% Redefines (sub)paragraphs to behave more like sections
\ifx\paragraph\undefined\else
\let\oldparagraph\paragraph
\renewcommand{\paragraph}[1]{\oldparagraph{#1}\mbox{}}
\fi
\ifx\subparagraph\undefined\else
\let\oldsubparagraph\subparagraph
\renewcommand{\subparagraph}[1]{\oldsubparagraph{#1}\mbox{}}
\fi

% set default figure placement to htbp
\makeatletter
\def\fps@figure{htbp}
\makeatother

\defaultfontfeatures{Scale=MatchLowercase,Ligatures={TeX}}
\setmainfont{Liberation Serif}
\setsansfont{Liberation Serif}
\setmonofont[SmallCapsFont={Liberation Serif}]{Liberation Serif}
\usepackage{hyperref}
\usepackage{multirow}
\usepackage{xcolor}
\definecolor{cvlinkcolor}{RGB}{6,125,233}
\hypersetup{colorlinks,linkcolor=cvlinkcolor,citecolor=cvlinkcolor,urlcolor=cvlinkcolor,bookmarksnumbered,bookmarks=false,pdfinfo={Title={Curriculum Vitae},Author={Nikolaos Tourvas},Creator={Lighton Phiri},Subject={Curriculum Vitae},Keywords={Curriculum Vitae},CreationDate={D:20150513232724}}}
\renewcommand\labelenumi{[\theenumi]}

\date{}

\begin{document}

\begin{table}[h]
%%%\centering
{\def\arraystretch{1.0}\tabcolsep=0pt
\begin{tabular}{p{0.42\linewidth}p{0.05\linewidth}p{0.35\linewidth}}

  \multirow{1}{*}{\LARGE \textbf{Νικόλαος Τουρβάς} | Βιογραφικό Σημείωμα} &  &  \\
  
  & & \\
  
  \multirow{1}{*}{\centering{\today}}&  &  \\

  \multirow{1}{*}{\centering{\href{https://github.com/nikostourvas/CV/blob/master/CV.pdf}{Ενημερωμένη έκδοση} }}&  &  \\

  & & \\ 
  
  
%  Ιωάννη Χρυσοστόμου 13 & \multicolumn{1}{r}{Researchgate:\;\;} & \multicolumn{1}{l}{\url{researchgate.net/profile/Nikolaos_Tourvas}} \\
  
  Τηλέφωνο: $+$30 6957686563 & \multicolumn{1}{r}{ORCID iD:\;\;} & \multicolumn{1}{l}{\href{https://orcid.org/0000-0002-0476-4468}{orcid.org/0000-0002-0476-4468}} \\
  
  Email: {\href{ntourvas@for.auth.gr}{ntourvas@for.auth.gr}&
  \multicolumn{1}{r}{Twitter:\;\;} & \multicolumn{1}{l}{\href{https://twitter.com/NTourvas}{@NTourvas}} \\
  
  & \multicolumn{1}{r}{Github:\;\;} & \multicolumn{1}{l}{\href{https://github.com/nikostourvas}{https://github.com/nikostourvas}} \\
  
\end{tabular}%
}
\end{table}


\subsection{Εκπαίδευση}\label{Εκπαίδευση}
\vspace{-3mm}

\textbf{Νοε 2018 -- σήμερα: ΠΜΣ Γενετική, Βελτίωση Φυτών και Παραγωγή Πολ/κού Υλικού, Τμήμα Γεωπονίας, ΑΠΘ}
\begin{itemize}
\setlength\itemsep{-0.5em}
\item \textit{Μεταπτυχιακή Διατριβή}: Μοριακή γενετική ποικιλότητα επιλεγμένων πληθυσμών αγριελιάς και σύγκριση με το γενετικό προφίλ καλλιεργούμενων ποικιλιών
\vspace{1mm}
\newline
\textit{Επιβλέπων}: καθηγητής Φίλιππος Α. Αραβανόπουλος \newline 
\phantom{1} \hspace{1.58cm} καθηγητής Αλέξιος Πολύδωρος
\end{itemize}

\vspace{2mm}

\textbf{Σεπτ 2012 -- Απρ 2018: Πτυχίο Δασολογίας και Φυσικού Περιβάλλοντος, ΑΠΘ, Βαθμός 7,86 - Λίαν Καλώς}
\begin{itemize}
%\vspace{-3mm}
\setlength\itemsep{-0.5em}
\item \textit{Πτυχιακή Διατριβή}: Συγκριτική ανάλυση μοριακής γενετικής ποικιλότητας
ώριμων ατόμων και αναγέννησης ελάτης στα πλαίσια
προγράμματος γενετικής παρακολούθησης
\vspace{1mm}
\newline
\textit{Επιβλέπων}: καθηγητής Φίλιππος Α. Αραβανόπουλος
\end{itemize}

\vspace{-3mm}

\subsection{Εργασιακή Εμπειρία σε ερευνητικά προγράμματα}\label{Εμπειρία}
\vspace{-3mm}
%\textbf{Μάι 2018 - Ιαν 2019: Συμμετοχή στο ερευνητικό πρόγραμμα ECOVARIETY} (Ανάδειξη τοπικών παραδοσιακών ποικιλιών και αυτοφυών οπωροφόρων δέντρων και θάµνων) - ΙΓΒΦΠ (ΕΛΓΟ-ΔΗΜΗΤΡΑ)

%\vspace{3mm}
%\begin{itemize}
%\vspace{-3mm}
%\setlength\itemsep{-0.6em}
%\item Ταυτοποίηση ποικιλιών αυτοφυών φυτικών ειδών με τη χρήση μοριακών δεικτών
%\item Εφαρµογή εργαλείων και λογισµικών χαρτογράφησης πληθυσµών
%\vspace{2mm}
%\newline
%\textit{Επιστημονικά υπεύθυνος}: Δρ. Ιωάννης Γανόπουλος, Εντεταλμένος Ερευνητής
%\end{itemize}

%\vspace{3mm}
\textbf{Ιουλ 2021 - Δεκ 2021}: GoCitrus: Χρήση σύγχρονων βιο-αναλυτικών μεθόδων στην διάσωση και αξιοποίηση των Ελληνικών ποικιλιών εσπεριδοειδών
\vspace{2mm}
\newline
\textit{Επιστημονικά υπεύθυνος}: καθηγητής Φίλιππος Αραβανόπουλος

\textbf{Φεβ 2021 - Μαρ 2021}: Ερευνητικό έργο "Γενετική έρευνα του πληθυσμού Ιταμου (\textit{Τaxus baccata}) του πανεπιστημιακού δάσους Ταξιάρχη με σκοπό τη γενετική
βελτίωση για την παραγωγή ταξόλης" - ΑΠΘ, Θεσσαλονίκη
\vspace{2mm}
\newline
\textit{Επιστημονικά υπεύθυνος}: καθηγητής Φίλιππος Αραβανόπουλος

\textbf{Νοε 2019 - Οκτ 2020}: Εμβληματική δράση "Οι Δρόμοι Της Ελιάς" Υποέργο 1: Γενετική και Γονιδωματική Ανάλυση - ΑΠΘ Θεσσαλονίκη
\vspace{2mm}
\newline
\textit{Επιστημονικά υπεύθυνος}: αναπληρωτής καθηγητής Αθανάσιος Μολασιώτης

\textbf{Νοε 2019 - Απρ 2020}: Εμβληματική δράση "Οι Δρόμοι Της Ελιάς" Υποέργο 1: Γενετική και Γονιδωματική Ανάλυση. Γενετικός Χαρακτηρισμός ελληνικών ποικιλιών ελιάς με χρήση μοριακών δεικτών SSR - Ινστιτούτο Γενετικής Βελτίωσης και Φυτογενετικών Πόρων, ΕΛΓΟ-ΔΗΜΗΤΡΑ, Θεσσαλονίκη
\vspace{2mm}
\newline
\textit{Επιστημονικά υπεύθυνος}: εντεταλμένος ερευνητής Ιωάννης Γανόπουλος

\textbf{Μαρ 2020 - Μαρ 2020}: Ερευνητικό έργο "Γενετική ταυτοποίηση ποικιλιών καρυδιάς" - ΑΠΘ Θεσσαλονίκη
\vspace{2mm}
\newline
\textit{Επιστημονικά υπεύθυνος}: καθηγητής Φίλιππος Αραβανόπουλος

\textbf{Ιουλ 2018 - Δεκ 2020}: Ερευνητικό έργο "LIFE για τη γενετική παρακολούθηση των ευρωπαικών δασών" - ΑΠΘ Θεσσαλονίκη
\vspace{2mm}
\newline
\textit{Επιστημονικά υπεύθυνος}: καθηγητής Φίλιππος Αραβανόπουλος

\textbf{Μαι 2018 - Ιουν 2018}: Ερευνητικό έργο "Informed: Ενοποιημένη Έρευνα για την Αντοχή και Διαχείριση Δασών στη Μεσόγειο" - ΑΠΘ Θεσσαλονίκη
\vspace{2mm}
\newline
\textit{Επιστημονικά υπεύθυνος}: καθηγητής Φίλιππος Αραβανόπουλος

\vspace{3mm}
\textbf{Απρ - Μάι 2016: Πρακτική άσκηση - Ινστιτούτο Εφαρμοσμένων Βιοεπιστημών - ΕΚΕΤΑ, Θεσσαλονίκη}


\vspace{3mm}
\begin{itemize}
\vspace{-3mm}
\setlength\itemsep{-0.6em}
\item Μοριακός χαρακτηρισμός ποικιλιών μήλου με χρήση μοριακών δεικτών SSR 
\item Εκχύλιση DNA από παλαιωμένη ξυλεία πεύκης και ταξινομικός προσδιορισμός βάσει της χλωροπλαστικής αλληλουχίας \textit{trnL} με τη μέθοδο HRM
%\end{itemize}
%\vspace{-2mm}
%\hspace{11mm}
\vspace{2mm}
\newline
\textit{Επιβλέπων, EKETA}: Δρ. Παναγιώτης Μαδέσης, Κύριος Ερευνητής
\newline
%\hspace{11mm}
\textit{Επιβλέπων, ΑΠΘ}: καθηγητής Φίλιππος Α. Αραβανόπουλος
\end{itemize}

%\newpage

\subsection{Εργαστηριακές δεξιότητες}\label{lab}
\begin{itemize}
\vspace{-3mm}
\setlength\itemsep{-0.6em}
\item Εκχύλιση DNA, Ποσοτικοποίηση νουκλεικών οξέων με φασματοφωτόμετρο
\item Multiplex / touchdown / gradient / nested PCR, qPCR - HRM
\item Fragment analysis (ABI SeqStudio platform - GeneMapper)
\end{itemize}


\subsection{Δεξιότητες ανάλυσης δεδομένων γενετικής πληθυσμών}\label{genetics}
%\vspace{-3mm}
%Λογισμικό: GenAlEx, FSTAT, Genepop, Arlequin, Structure, MEGA, POPTREE2, NeEstimator, Bottleneck, R (πολλαπλά πακέτα), κ.α.
\begin{itemize}
\vspace{-3mm}
\setlength\itemsep{-0.6em}
\item Ανάλυση πρωταρχικών δεδομένων από γενετικό αναλυτή τύπου ΑΒΙ (GeneMapper)
\item Βασικές παράμετροι ποικιλότητας (GenAlEx, FSTAT, πακέτα R: poppr, genepop, pegas κ.α.)
\item Γενετική διαφοροποίηση: AMOVA, Ομαδοποίηση βάσει πολυμεταβλητών αναλύσεων (PCA, PCoA, CA, DAPC) - (πακέτα R: adegenet, hierfstat, mmod, ade4 κ.α.)
\item Ομαδοποίηση κατά Bayes - (Structure)
\item Φυλογενετική ανάλυση: UPGMA, Neighbor-Joining - (MEGA, POPTREE2, πακέτα R: ape, poppr)
\item Outlier analysis: Bayescan, Fdist2, Outflank, PCAdapt
\item Εκτίμηση δραστικού μεγέθους πληθυσμού, γενετικών στενωπών, μηδενικών αλληλομόρφων - (NeEstimator, Bottleneck, ML-NullFreq, R: strataG)
\item Γενετική τοπίου: Mantel test IBD - (GenAlEx, πακέτο R: adegenet)
%\item Γενετική χαρτογράφηση με τη χρήση μοριακών δεικτών

%\item FST outliers
\end{itemize}

\subsection{Δεξιότητες χρήσης Η/Υ}\label{it}
\begin{itemize}
\vspace{-3mm}
\setlength\itemsep{-0.6em}
\item Προγραμματισμός: R
%- δείγματα στο \href{https://github.com/nikostourvas}{github})
\item Λειτουργικά συστήματα: Windows (συμπ. Microsoft Office), Linux
\item Κατασκευή γραφημάτων: ggplot2 %\href{https://ggplot2.tidyverse.org/}{ggplot2}
\item Συστήματα στοιχειοθεσίας: Markdown, \LaTeX
\item Σύστημα ελέγχου εκδόσεων: git
%\item Παρουσίαση χωρικών δεδομένων: ArcGIS, QGIS
%\item Χωρική ανάλυση οικοσυστημικών υπηρεσιών (πλατφόρμα \href{https://naturalcapitalproject.stanford.edu/invest/}{InVEST})
\item Επαναληψιμότητα - Διαμοιρασμός αναλύσεων: Docker %\href{https://arxiv.org/pdf/1410.0846.pdf}{Docker}
\end{itemize}

\subsection{Λογισμικό}\label{software}
\begin{itemize}
\vspace{-3mm}
\setlength\itemsep{-0.6em}

\item Συντήρηση εικονικού υπολογιστικού περιβάλλοντος Docker για τη διενέργεια στατιστικών αναλύσεων δεδομένων γενετικής πληθυσμών 
\href{https://github.com/nikostourvas/forest_popgen}{forest\_popgen}

\item Πακέτο λογισμικού R \href{https://github.com/nikostourvas/PopGenUtils}{PopGenUtils} (υπό ανάπτυξη)

\end{itemize}

\subsection{Επιμόρφωση}\label{courses}
\begin{itemize}
\vspace{-3mm}
\setlength\itemsep{-0.6em}

\item Εντατικό μάθημα βραχείας διάρκειας:
COST G-BIKE Training School: "Genomic tools for conservation: a practitioner’s guide", La Valletta (Malta), 20-22 January 2020

\item Εντατικό μάθημα βραχείας διάρκειας: "Concepts, Methods and Tools for an Integrated Approach of Resilience in Mediterranean Forests", Zaragoza (Spain), 12-16 December 2016
%\item Διεθνές εκπαιδευτικό εργαστήριο (workshop): "Informed 2nd Modelling workshop", Madrid (Spain), 23-25 November 2016

\end{itemize}

\subsection{Γλωσσικές Δεξιότητες}\label{Γλώσσες}
\begin{itemize}
\vspace{-3mm}
% * <nikostourvas@gmail.com> 2018-08-12T12:16:45.474Z:
%
% ^.
\setlength\itemsep{-0.6em}
\item \textbf{Αγγλική γλώσσα} - Άριστη γνώση - C2 - University of Michigan

\item \textbf{Γερμανική γλώσσα} - Καλή γνώση - Β2 - Goethe Institut
\end{itemize}



\subsection{Δημοσιεύσεις}\label{publications}

\subsubsection{Επιστημονικά περιοδικά με συντελεστή απήχησης}\label{journals}
\vspace{-3mm}
\begin{enumerate}

\leftskip-0.07in

\item El Chami, M.A., \textbf{Tourvas, N.}, Kazakis, G., Kalaitzis, P., Aravanopoulos, F.A. (2021) 'Genetic characterisation of chestnut cultivars in Crete' \textit{Forests}. p. 12. doi: \href{https://doi.org/10.3390/f12121659}{10.3390/ 
f12121659}

\item Boutsika, A., Sarrou E., Cook C., Mellidou, I., Avramidou, E., Angelic, A., Martens, S., Rallia, P., Letsiou S., Selinia, A., Grigoriadis, I., \textbf{Tourvas, N.}, Kadoglidou, K., Kalivas, A., Maloupa, E., Xanthopoulou, A., Ganopoulos, I. (2021) 'Evaluation of parsley (\textit{Petroselinum crispum}) germplasm diversity from the Greek Gene Bank using morphological, molecular and metabolic markers', \textit{Industrial Crops \& Products}. doi: \href{https://doi.org/10.1016/j.indcrop.2021.113767}{10.1016/j.indcrop.2021.113767}

\item Ganopoulos, I., \textbf{Tourvas, N.}, Xanthopoulou, A., Aravanopoulos, F. A., Avramidou, E. V., Zambounis, A., Tsaftaris, A., Madesis, P., Sotiropoulos, T. and Koutinas, N. (2018) ‘Phenotypic and molecular characterization of apple (\textit{Malus} x \textit{domestica} Borkh) genetic resources in Greece’, \textit{Scientia Agricola}, 75(6), pp. 509–518. doi: \href{http://dx.doi.org/10.1590/1678-992x-2016-0499}{10.1590/1678-992x-2016-0499}

\end{enumerate}


\subsubsection{Πρακτικά Συνεδρίων}\label{conferences}

\vspace{-3mm}
\begin{enumerate}
\leftskip-0.13in
	\setcounter{enumi}{3}
	
\leftskip-0.07in  

\leftskip-0.07in  

\item Jansen, Simon; Acar, Pelin; Aravanopoulos, Filippos; Çiftçi, Asiye; Değirmenci, Funda Özdemir; Duyar, Kürşad; İdman, Özlem Mavi; Kansu, Çiğdem; Kaya, 
Zeki; Kleinschmit, Jörg; Leigh, Deborah M.; Lyrou, Fani G.; Rellstab, Christian; 
Semizer-Cuming, Devrim; \textbf{Tourvas, Nikolaos}; Neophytou, Charalambos (2021) 'Is the grass always greener on the other side? Identifying seed sources for oak forests in a changing climate' Genomics and Adaptation in Forest Ecosystems. EvolTree Conference 2021, 14–17 September 2021, Birmensdorf, Switzerland. \href{http://doi.org/10.16904/envidat.243}{10.16904/envidat.243}

\item \textbf{Tourvas, N.}, Westergren, M., Fussi,  B.,  Bajc,  M.,  Malliarou  E.,  Kavaliauskas,  D.,  Farsakoglou A.M., Kiourtsis F., Kraigher, H., Aravanopoulos F.A. (2020) Minimum Requirements for Genetic Monitoring: A Proposed Workflow. Forest science for future forests: forest genetic monitoring and biodiversity in changing environments: LIFEGENMON final conference: 21 - 25 of September 2020, Ljubljana, Slovenia: Silva Slovenica, Gozdarski inštitut Slovenije. 
\href{http://doi.org/10.20315/SFS.162}{10.20315/SFS.162}
\newline
YouTube: \href{https://www.youtube.com/watch?v=Bx2IJguytPk}{https://www.youtube.com/watch?v=Bx2IJguytPk}

\item \textbf{Tourvas, N.}, Malliarou, E., Westergren, M., Fussi, B., Bajc, M., Kavaliauskas, D., Barbas, E., Alizoti, P., Kiourtsis, F., Kraigher, H., Aravanopoulos F.A., (2020) Genetic monitoring in the hybridogenous fir (\textit{Abies borisii-regis}): Interpretation of the first temporal and intergenerational comparison using SSR genetic markers. Forest science for future forests: forest genetic monitoring and biodiversity in changing environments: LIFEGENMON final conference: 21 - 25 of September 2020, Ljubljana, Slovenia: Silva Slovenica, Gozdarski inštitut Slovenije. \href{http://doi.org/10.20315/SFS.162}{10.20315/SFS.162}
\newline
YouTube: \href{https://www.youtube.com/watch?v=G10XtjA-XK0}{https://www.youtube.com/watch?v=G10XtjA-XK0}

%\item Aravanopoulos, F.A., Malliarou, E., \textbf{Tourvas, N.}, Avramidou, E.V., Ganopoulos, I., Farsakoglou, A.-M., Barbas, E., Alizoti, P., Westergren, M., Bajc, M., Kavaliauskas, D., Kiourtsis, F., Fussi, B., Kraigher, H., (2020). Population genetic and epigenetic monitoring for evaluating genetic resources across time and space. Presented at the GenTree Final Conference: \textit{Genetics to the Rescue - Managing Forests Sustainably in a Changing World}, Avignon, France.

\item \textbf{Tourvas, N.}, \& Aravanopoulos, F. A. (2019). \textit{Minimum Requirements for Genetic Monitoring: A Proposed Workflow.} Presented at the Hellenic Bioinformatics 12, Precision Medicine, Biodiversity \& Genome Biology, Heraklion, Greece.

\item Aravanopoulos, F. A., \textbf{Tourvas, N.}, Malliarou, E., Avramidou, E., Farsakoglou, A. M., Alizoti, P. G., … Kraigher, H. (2019, October 9). \textit{Interim Genetic Monitoring Application in the Presence of Incidental Data: Theory and Application}. Presented at the XXV IUFRO World Congress, Forest Research and Cooperation for Sustainable Development, Curitiba, PR, Brazil. \href{https://doi.org/10.4336/2019.pfb.39e201902043}{10.4336/2019.pfb.39e201902043}

%\item {\textbf{Tourvas, N.}, M. Westergren, M. Bajc, N. Dovč, D. Finžgar, B. Fussi, D. Kavaliauskas, E. Malliarou, H. Kraigher, F. A. Aravanopoulos (2019) ‘Optimal Sampling for Accurate Estimation of Genetic Diversity: Preliminary Results’, in \textit{AForGeN Alpine Forest Genomics Network.} Mount Ventoux, France.}

%\item {Ballian, D., Westergren, M., Bozic, G., Bajc, M., Finzgar, D., Fussi, B., Konnert, M., Kavaliauskas, D., Aravanopoulos, F.A., Alizoti, P., Malliarou, E., \textbf{Tourvas, N.}, Kiourtsis, F., Breznikar, A., Rantasa, B., Kraigher, H. (2019). Implementation of a concept of Forest genetic monitoring on a transect from Germany to Greece. Presented at the \textit{Prospects for fir management in a changeable Environment}, Krakow, Polska. }

\item {\textbf{Τουρβάς, Ν.}, Αραβανόπουλος Φ. Α. (2018) ‘Γενετική θεώρηση των πρακτικών αναδάσωσης για τη μεγιστοποίηση της ανάπαλσης των δασικών οικοσυστημάτων’, Πρακτικά \textit{17ο Πανελλήνιο Συνέδριο της Ελληνικής Επιστημονικής Εταιρείας Γενετικής Βελτίωσης Φυτών.} Πάτρα.}

\leftskip-0.07in  
\item {\textbf{Tourvas, N.} and Aravanopoulos, F. A. (2018) ‘Genetic aspects of reforestation practices for the maximization of forest ecosystem resilience’, in \textit{International Conference Reforestation Challenges.} Belgrade, Serbia, p. 82.}

\leftskip-0.07in  
%\item {Aravanopoulos, F. A., Westergren, M., \textbf{Tourvas, N.}, Fussi, B., Finzgar, D., Bajc, M., Kavaliauskas, D., Malliarou, E., Kiourtsis, F. and Kraigher, H. (2018) ‘Genetic monitoring of hybridogenus populations: an analysis of \textit{Abies borisii-regis}, a hybrid between \textit{Abies alba }and the Greek endemic \textit{Abies cephalonica}’, in \textit{AForGeN Alpine Forest Genomics Network.} Kranjska Gora, Slovenia.}

%\item {\textbf{Tourvas, N.} and Aravanopoulos, F. A. (2018) ‘Genetic analysis and genetic monitoring of fir at different spatial scales’, in \textit{5th Students Conference: Organic Farming, Remote Sensing of Environment.} Thessaloniki: Faculty of Agriculture, Forestry and Natural Environment.

\leftskip-0.07in  
\item {Aravanopoulos, F. A., Avramidou, E. V., Malliarou, E., \textbf{Tourvas, N.}, Kotina, V., Korompoki, I., Ganopoulos, I., Alizoti, P., Barbas, E., Bekiaroglou, P., Hasilidis, P., Roussakis, G., Kiourtsis, F. and Fragiskakis, N. (2017) ‘Forest genetic monitoring (FGM) applied: first results from two FGM sites in Greece’, in \textit{Forest Genetics 2017: Health and Productivity under Changing Environments. A Joint Meeting of WFGA and CFGA,} University of Alberta, Edmonton.}

\leftskip-0.07in  
\item {Aravanopoulos, F. Α., \textbf{Tourvas, N.}, Avramidou., E., Alizoti, P., Barbas, E. (2017). The European research project INFORMED and the research in Greece on the resilience of Mediterranean forests. In: Proc. \textit{18th Pan-Hellenic Forest Science Society Conference,} Edessa, pp. 429-433}
 
\leftskip-0.07in  
\item {\textbf{Τουρβάς, Ν.}, Μαλλιαρού, E., Αβραμίδου, Ε. Β. Αραβανόπουλος, Φ. Α. (2017) ‘Συγκριτική ανάλυση μοριακής γενετικής ποικιλότητας ώριμων ατόμων και αναγέννησης ελάτης στα πλαίσια προγράμματος γενετικής παρακολούθησης’, Πρακτικά \textit{18ο Πανελλήνιο Δασολογικό Συνέδριο.} Έδεσσα.}

\leftskip-0.07in  
\item {\textbf{Tourvas, N.}, Malliarou, E., Avramidou, E. V. and Aravanopoulos, F. A. (2016) ‘Genetic Monitoring in fir: Initial results from comparative analysis of molecular genetic diversity of adult and regeneration individuals’, in \textit{8th Hellenic Ecological Society Conference.} Thessaloniki.}

\leftskip-0.07in  
\item {Aravanopoulos, F. A., Avramidou, E. V., Malliarou, E., \textbf{Tourvas, N.}, Ganopoulos, I., Alizoti, P., Barbas, E., Bekiaroglou, P., Hasilidis, P., Roussakis, G., Kiourtsis, F., Fragiskakis, N., Westergren, M., Fussi, B., Konnert, M. and Kraigher, H. (2016) ‘Life for forest genetic monitoring: an opportunity to monitor marginal forest tree populations in Greece’, in \textit{COST Action FP1202 MaP-FGR MC final meeting and COST FP1202, EUFORGEN, IUFRO WG 20213 joint final Conference on ‘Marginal and peripheral tree populations: a key genetic resource for European forests’,} Arezzo, Italy.}

\leftskip-0.07in  
\item {\textbf{Τουρβάς, Ν.}, Αβραμίδου, Ε. Β., Αραβανόπουλος Φ. Α. (2016) ‘Πρώτες συγκριτικές αναλύσεις ατόμων ελάτης (\textit{Abies borisii-regis}
) σε διαφορετικό στάδιο ανάπτυξης με τη χρήση μικροδορυφόρων SSR’, Πρακτικά \textit{16ο Πανελλήνιο Συνέδριο της Ελληνικής Επιστημονικής Εταιρείας Γενετικής Βελτίωσης Φυτών.} Φλώρινα.}

\end{enumerate}


\subsubsection{Κεφάλαια βιβλίων διεθνών εκδόσεων}\label{book-chapters}

\begin{enumerate}
\setcounter{enumi}{16}
\leftskip-0.07in
\vspace{-3mm}

\item Aravanopoulos F.A., Alizoti P.G., \textbf{Tourvas N.}, Malliarou E., Avramidou E.V., Korompoki I.V., Kotina V.M., Barbas V. ,  Farsakoglou A.M. (2019). Overview on forest genetic monitoring including case studies on FGM for two species from Greece. In: Sijacic-Nikolic M., Milovanovic J. \& M. Nonic (eds). \textit{Forests of Southeast Europe Under a Changing Climate: Conservation of Genetic Resources (pp. 401–407)}. Springer


\item Aravanopoulos F.A., Alizoti P.G., Farsakoglou A.M., Malliarou E., Avramidou E.V. , \textbf{Tourvas N.}  (2019). State of biodiversity and forest genetic resources in Greece in relation to conservation. In: Sijacic-Nikolic M., Milovanovic J. \& M. Nonic (eds). \textit{Forests of Southeast Europe Under a Changing Climate: Conservation of Genetic Resources (pp. 73-83)}. Springer

\item Bajc, M., Aravanopoulos, F., Westergren, M., Fussi, B., Kavaliauskas, D., Alizoti, P., … Kraigher, H. (Eds.). (2020). Manual for forest genetic monitoring. Ljubljana: Slovenian Forestry Institute, Silva Slovenica Publishing Centre. http://doi.org/10.20315/SFS.167
\end{enumerate}

%\subsubsection{Υπό συγγραφή}\label{in_preparation}

%\begin{enumerate}
%	\setcounter{enumi}{13}
%\leftskip-0.07in
%\vspace{-3mm}

%\item Westergren, M., \textbf{Tourvas, N.}, Finzgar, D., Kavaliauskas, D., Bajc, M., Malliarou, E., Fussi, B., Kraigher, H., Aravanopoulos, F. A. 'Estimation of sample size for accurate estimation of genetic diversity'

%\end{enumerate}

\subsection{Δραστηριότητες επικοινωνίας με το ευρύ κοινό}\label{public_outreach}
\begin{itemize}
\vspace{-3mm}
\setlength\itemsep{-0.6em}
\item Ξεναγήσεις στο μουσείο φυσικής ιστορίας του τμήματος Δασολογίας και Φυσικού Περιβάλλοντος σε μαθητές πρωτοβάθμιας / δευτεροβάθμιας εκπαίδευσης (2014 - 2016) και συμμετοχή στην πρωτοβουλία \href{https://www.auth.gr/sites/default/files/web_final.pdf}{ΑΠΘ την Κυριακή} (2015)
\vspace{2mm}
\newline
\textit{Υπεύθυνος}: Δρ. Νικόλαος Παραλυκίδης, Εργαστήριο Άγριας Πανίδας και Ιχθυοπονίας Γλυκέων Υδάτων, ΑΠΘ
\end{itemize}


%\newpage

%\subsection{References}\label{references}

%Available on request.

\end{document}